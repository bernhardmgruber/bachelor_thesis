\section{Summary and conclusion}

OpenCL is a free and open standard for general purpose parallel programming across various hardware devices. It is maintained by the Khronos Group and implemented by hardware vendors like NVIDIA, AMD or Intel. OpenCL is natively available for C and C++ although many wrappers for other languages exist. To use OpenCL in a C/C++ application an SDK is needed which contains the required header files (also available from Khronos) and static libraries. For running an application featuring OpenCL a corresponding driver has to be installed offering the API functions as well as a compiler to create kernels. Furthermore, one or more appropriate hardware devices supporting OpenCL are necessary and may even be used together.

The hardware architectures for which OpenCL can be used are quite different, especially concerning the differences between CPUs and GPUs.
Modern consumer CPUs feature a small number of independent, high power cores (mostly four or eight) which can be used independently by an application using threads. Synchronization between the threads is easy and threads may be spawned at any time. High throughput can be achieved by distributing the work on all available cores and make use of vector instruction sets like SSE or AVX. Memory is provided as a large block behind a hierarchical cache system. \\
GPUs however employ a massive amount of less powerful cores (up to several thousands) packed together in multiprocessors. Work is processed in smaller work groups and larger n-dimensional ranges which often occupy hundreds or thousands of threads (which often have to be a multiple of hardware specific resources) executing the same code step by step. Synchronization is very limited and the amount of threads used has to be predetermined every time work is offloaded to the GPU. High throughput is achieved by utilizing all components of the GPU (ALU, read and write units, etc.) as much as possible. Branching and synchronization should be avoided and vector types/operations should be used wherever possible to maximize the ALUs throughput. The large global memory is cached but very slow and sensitive concerning access patterns. Local memory is available as programmers-controlled cache but again suffers from misaligned accesses (bank conflicts). However, GPUs can achieve a much higher computational throughput than CPUs when used correctly.

Using OpenCL in an application starts by choosing an available platform/implementation and device. Additionally, a context and a command queue have to be created. Kernels are mostly created directly from their source code which has to be provided at run time. The source code is then compiled and linked into a program for the selected device and the kernels can be queried by name. To provide larger input and to retrieve output from a kernel, buffer objects have to be created. These may be read from and written to, either asynchronously or synchronously to the main application, by enqueuing operations on the command queue. Kernels are also enqueued and executed asynchronously after their arguments have been set.

To show the strengths and weaknesses of OpenCL and GPGPU computing on a few real world examples, three kinds of problems have been chosen for which various approaches have been implemented and benchmarked. \\
At the beginning, multiplying two square matrices was ported to OpenCL. The CPU implementation is fairly trivial but performs badly. Fortunately, highly tuned BLAS libraries exist which can do the multiplication a lot faster on the CPU. However, the first and again quite simple OpenCL implementation already beat the efficient BLAS sgemm routine. With some optimizations, making the code more complicated, we could push performance even further achieving a gigantic speedup of 36.
By the example of matrix multiplication we also took a look into performance analysis using profiler information from AMD's CodeXL. \\
The second problem tackled was the all-prefix sum, a linear algorithm. The CPU implementation can be coded in a minute using a simple loop in a few lines of code. A naive, also quite easy to understand GPU approach performed fairly poor. By trying a more sophisticated tree-based reduce and scatter algorithm, small improvements could be made. Adding local memory and finally also using vector types and applying the algorithm on several layers in the end showed significant cut-downs on the run time (at the cost of hundreds of lines of code). However, the runtime of the CPU could not be beaten. \\
The last implementation chapter focused one of the most famous topics in computer science, sorting. The two standard library routines \lstinline!qsort! and \lstinline!std::sort! have been benchmarked together with a radix sort implementation, the latter delivering amazing results. For sorting on the GPU, sorting networks have been introduced because they are easy to parallelize. In particular the bitonic sorter has been implemented and optimized. Despite the nasty input size restriction, the GPU implementation runs more than twice as fast as the comparison based variants on the CPU, which can be clearly seen as success. Radix sort however suffers from the GPUs highly parallel nature. Despite several optimizations it could not catch up and remains far behind the CPU implementation.

In conclusion it can be said that GPUs are different from CPUs in the way they process work. This originates from their quite dissimilar hardware architectures as we discussed in the beginning. But not only the hardware itself, especially also the way algorithms have to be designed and programs are written differs largely from traditional single-thread-orientated CPU algorithms. Although it is already quite common to assign several tasks to several CPU cores (e.g. GUI and work thread), bringing parallelism down into the deeps of small routines is still far from being common. When a program is designed to be run on massively parallel hardware we tend to spend more time on fighting with the peculiarities of the hardware than thinking about the actual problem we try to solve. This was one of the hardest challenges when implementing scan. The scan operation imposes high data dependency between the processed elements and therefore makes it perfectly suitable for as few processing units as possible. The GPU however benefits from a maximum of parallelism, on data and instruction level. Most of the time required to develop an efficient GPU scan was spent on trying to get enough parallelism on the cost of minimal redundancy in additions. Although the tree based approach worked well in the end, a lot of computing power is wasted during the up-sweep and down-sweep phase inside each work group. The same problem affected the radix sort GPU implementation. The per work item histogram was only required because the GPU does not ensure the order in which work items are processed. And to improve memory reading performance for the scan step by aligning the histogram buffer for coalesced access, the scattered, unblocked writes of the histograms at the end of the permute step also teared down performance; Because GPUs profit enormously from reading and writing larger, consecutive blocks of memory. On the CPU, apart from worse caching, scattered accesses is less of a problem. The final argument, that should arise from this observations, is that a programmer has to worry about a lot more different stuff than he has when coding for a processor. Imagine a Java or C\# programmer who is used to object orientated design, dynamic memory allocation, huge standard libraries, containers and other complex data structures and all that candy of modern languages, when he finds himself porting one of his well-written, readable algorithms to the GPU where he has to break everything down into well-layouted arrays, think about the vectorizability of the produced instructions, repeatedly check the API documentation for a small number of built in function that might be of use and hunt down performance issues to stalled memory reads and bank conflicts. GPGPU computing is far different from modern software design and programming. Although the API is easily understood and the kernel language quickly learned, it takes months and probably years of practice to get a good feeling for designing GPU accelerated algorithms. GPUs are technological marvels offering immense computational power if used correctly. But it takes time to learn and study them in order to unfold their full potential.

Furthermore, it has to be said that not all algorithms fit the requirements of a GPU. In fact, only a small subset really benefits from the massively parallel hardware. In most cases it is easier to focus on spreading a workload on several cores of the main processor instead of dealing with thousands of threads. Also the required memory transfer to and from the GPU is a crucial factor when deciding to offload a piece of work or not. If an algorithm does not fit the GPU one should not execute it there. \\
However, the small subset of algorithms which is often parallel in its nature may benefit tremendously from graphics hardware. We have discussed such a kind of algorithm on the example of multiplying matrices. The final speedup of 36 is amazing and shows the potential of GPU hardware when used at the right place.

So when should OpenCL be used? Apart from having the right kind of algorithm to take advantage of GPGPU acceleration, also time and budget play an important role. Developing an algorithm for the GPU can be tedious and cumbersome and therefore time consuming. Furthermore, GPUs also impose a higher risk of failing to achieve the desired performance boost. The scan algorithms with optimizations may have taken two weeks to develop with the final outcome that the CPU variant is still faster. Experimenting with GPUs in a real world project should therefore be well considered, also concerning the aspect of system requirements. From a developers perspective appropriate GPUs have to be available for implementing and testing and additional tools have to be installed for debugging and profiling. Concerning customers, appropriate hardware has to be available, drivers have to be installed and maybe the case of no available hardware or driver has to be handled (fallback CPU implementation). Talking about tools, the currently available suits from AMD (CodeXL), NVIDIA (Nsight, Visual Profiler) and Intel (SDK for OpenCL Applications) already provide decent help when developing software for their respective GPUs. However, they are still quite away from being as suitable and feature-rich as corresponding CPU debugger integration and profilers. Last but not least also maintainability and reusability play a vital role in modern software companies. Both aspects suffer in most cases as OpenCL algorithms tend to be optimized strongly to a specific kind of hardware and problem.

Finally we have to see how long GPGPU computing will be done in this fashion. With the increasing power of on-chip GPUs on modern Intel processors, using GPUs for smaller tasks might become more attractive as no memory transfers will be required. AMD is also working intensively on a new generation of processing devices called Accelerated Processing Units (APU). By using a so-called heterogeneous system architecture (HSA) GPU and CPU are combined tightly on a single chip. The GPU will have direct access to the system's main memory with cache coherence to the CPU using the same address space as the CPU (theoretically allowing pointers to be passed). Furthermore, the GPU will allow multitasking via context switches between running tasks. As a result, long-running calculations will no longer freeze the display.
However, in another corner of the high performance hardware sector is Intel heavily working on their new Many Integrated Core (MIC) architecture. The Intel Xeon Phi (previously named Knights Corner) is a co-processor card with up to 61 individual Intel processors and up to 16 GiB dedicated on-card memory. From an OpenCL programmers perspective the Xeon Phi is equally programmed as a GPU. But the card is composed of modified x86 processors and can therefore also execute conventional software written in traditional languages. As a result, language integration is also easy. In fact, Intel's compiler already offers appropriate \lstinline!#pragma!s to offload parts of a C++ program to the co-processor card in a similar fashion as OpenMP. In addition, almost all drawbacks of GPUs vanish as all threads (four cores on each processor) can be fully synchronized, memory can be allocated dynamically on demand and all existing language features including libraries can be used. However, a Xeon Phi is still financially out of reach for consumers and still more expensive than decent GPUs with prices among several thousand dollars. But as time passes by they may become affordable. Intel already announced that the next generation of the Xeon Phi (codename Knights Landing) will be capable of being used as the systems main processor. So maybe in ten years we find ourselves with a hundred main processors in our notebooks capable of handling even computationally expensive software with ease. Who needs GPU computing then?
