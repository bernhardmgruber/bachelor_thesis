\section{Summary and conclusion}

OpenCL is a free and open standard for general purpose parallel programming across various hardware devices. It is maintained by the Khronos Group and implemented by hardware vendors like NVIDIA, AMD or Intel. OpenCL is natively available for C and C++ although many wrappers for other languages exist. To use OpenCL in a C/C++ application an SDK is needed which contains the required header files (also available from Khronos) and static libraries. For running an application featuring OpenCL a corresponding driver has to be installed offering the API functions as well as a compiler to create kernels. Furthermore one or more appropriate hardware devices supporting OpenCL are necessary and may even be used together.

The hardware architectures for which OpenCL can be used are quite different, especially concerning the differences between CPUs and GPUs.
Modern consumer CPUs feature a small number of independent, high power cores (mostly four or eight) which can be used independently by an application using threads. Synchronization between the threads is easy and threads may be spawned at any time. High throughput can be achieved by distributing the work on all available cores and make use of vector instruction sets like SSE or AVX. Memory is provided as a large block behind a hierarchical cache system. \\
GPUs however employ a massive amount of less powerful cores (up to several thousands) packed together in multiprocessors. Work is processed in smaller work groups and larger n-dimensional ranges which often occupy hundreds of threads (which often have to be a multiple of hardware specific resources) executing the same code step by step. Synchronization is very limited and the amount of threads used has to be predetermined every time work is offloaded to the GPU. High throughput is achieved by utilizing all components of the GPU (ALU, read and write units, etc.) as much as possible. Branching and synchronization should be avoided and vector types/operations should be used wherever possible to maximize the ALUs throughput. The large global memory is cached but very slow and sensitive concerning access patterns. Local memory is available as programmers-controlled cache but again suffers from misaligned accesses (bank conflicts). However, GPUs can achieve a much higher computational throughput than CPUs when used correctly.

Using OpenCL in an application starts by choosing an available platform/implementation and device. Additionally, a context and a command queue have to be created. Kernels are mostly created directly from their source code which has to be provided at run time. Therefore, the source code is compiled and linked into a program for the selected device and the kernels can be queried by name. To provide larger input and to retrieve output from a kernel, buffer objects have to be created. These may be read from and written to either asynchronously or synchronously to the main application by enqueuing operations on the command queue. Kernels are also enqueued and executed asynchronously after their arguments have been set.

To show the strengths and weaknesses of OpenCL and GPGPU computing on a few real world examples, three kinds of problems have been chosen for which various approaches have been implemented and benchmarked. \\
At the beginning, multiplying two square matrices was ported to OpenCL. The CPU implementation is fairly trivial but performs badly. Fortunately, highly tuned BLAS libraries exist which can do the multiplication a lot faster on the CPU. However, the first and again quite simple OpenCL implementation already beats the efficient BLAS sgemm routine. With some optimizations making the code more complicated we could push performance even further achieving a gigantic speedup of 36.
At the example of matrix multiplication we also took a look into performance analysis using profiler information from AMD's CodeXL.
The second problem tackled was the all-prefix sum, a linear algorithm. The CPU implementation can be coded in a minute using a simple loop in a few lines of code. A naive, also quite easy to understand approach performed fairly poor. By trying a more sophisticated tree-based reduce and scatter algorithm, small improvements could be made. Adding local memory and finally also using vector types and applying the algorithm on several layers in the end showed significant cut-downs on the run time (at the cost of hundreds of lines of code). However, the runtime of the CPU could not be beaten.
The last implementation chapter focused one of the most famous topics in computer science, sorting. The two standard library routines \lstinline!qsort()! and \lstinline!std::sort()! have been benchmarked together with a radix sort implementation, the latter delivering amazing results. For sorting on the GPU sorting networks have been introduced as easy to parallelize. In particular the bitonic sorter has been implemented and optimized. Despite the nasty input size restriction, the GPU implementation runs more than twice as fast as the comparison based variants on the CPU, which can be clearly seen as success. Radix sort however suffers from the GPUs highly parallel nature. Despite several optimizations it could not catch up and remains far behind the CPU implementation.


where can OpenCL be used?
when should it be used, when not?
required development time
flexibility of traditional implementations vs. OpenCL applications
maintainability?

\subsection{Technological advancements}
GPU and CPU on the same chip (Sandy Bridge)
MIC