\section{OpenCL}

\subsection{History of GPGPU Computing}
The development of graphics hardware

shader programming (as first possibility to program for GPUs, focus on graphic calculations, hence the name shader)

Using shaders for GPGPU programming ()

GPGPU Technologies
Cg, CUDA, OpenCL, DirectCompute

\subsection{What is OpenCL?}
open industry standard
Khronos
Vendors
Components (API, OpenCL C Language, Compiler, ...)

\subsection{Areas of application}
OpenCL
GPGPU computing
WebCL

\subsection{Requirements}
GPU and CPU Hardware
Drivers
SDK (Header and Libraries)
Bindings for other programming languages

\subsection{Hardware architectures}

GPU /CPU hardware
similarities/differences

\subsection{Hardware abstraction}
platforms, devices, command queues, programs, kernels, buffers, ...

\subsection{Memory model}

\subsubsection{Types of memory}
different types of memory on the GPU
global local constant private
performance
how do we access them?

\subsubsection{General design considerations}
coalesced memory access
local memory as programmable cache (and method of sync)
bench conflicts

\subsubsection{Data transfer}
read, write buffers
sync, async

\subsection{Kernel execution}
what is a kernel?

how are kernels executed?
work groups, work items
hardware threads (synchronization)

choosing the right work group size

\subsection{Structure of an OpenCL program}

with pseudocode example

