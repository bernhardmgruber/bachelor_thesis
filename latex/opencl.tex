\section{OpenCL}

\subsection{What is OpenCL?}
OpenCL is specified by the Khronos Group in the OpenCL 1.2 Specification as follows:

\begin{quotation}
OpenCL (Open Computing Language) is an open royalty-free standard for general purpose
parallel programming across CPUs, GPUs and other processors, giving software developers
portable and efficient access to the power of these heterogeneous processing platforms. \cite{opencl_spec}
\end{quotation}

The Khronos Group is an industry consortium who maintains OpenCL as an open standard. This means that the Khronos Group offers a specification of the OpenCL API and a detailed description about its functionality and behavior for free. This specification can be downloaded at their website \cite{opencl_spec}. Maintaining the OpenCL standard consequently means, that the Khronos Group neither provides software development kits (SDKs), drivers that implement OpenCL nor hardware that it can make use of. These concerns are subject to other companies called vendors, which are typically hardware manufacturers providing necessary developing resources and include OpenCL support in their drivers. Examples of such companies are the two famous graphic card vendors NVIDIA and AMD as well as the renowned processor manufacturer Intel.


Components (API, OpenCL C Language, Compiler, ...)

\subsection{Areas of application}
OpenCL
GPGPU computing
WebCL

\subsection{Requirements}
GPU and CPU Hardware
Drivers
SDK (Header and Libraries)
Bindings for other programming languages

\subsection{Hardware architectures}

GPU /CPU hardware
similarities/differences

\subsection{Hardware abstraction}
platforms, devices, command queues, programs, kernels, buffers, ...

\subsection{Memory model}

\subsubsection{Types of memory}
different types of memory on the GPU
global local constant private
texture, pinned memory
performance
how do we access them?

\subsubsection{General design considerations}
coalesced memory access
local memory as programmable cache (and method of sync)
bench conflicts

\subsubsection{Data transfer}
read, write buffers
sync, async

\subsection{Kernel execution}
what is a kernel?

how are kernels executed?
work groups, work items
hardware threads (synchronization)

choosing the right work group size

\subsection{Structure of an OpenCL program}

with pseudocode example

