\section{Summary and conclusion}
\label{sec:summary}

The RISC Software GmbH is a limited liability company and part of the software park Hagenberg. It focuses on the practical application of research done in the corresponding RISC institute. In mid 2011, the project Enlight was started with financial help by the governmental Regio 13 program. The goal of Enlight is to create a ray casting solution for interactive visualization of complex geometries consisting. Subtractive manufacturing is the main inspiration for the project. At the start of the internship in April 2013, most functionality was already implemented. The goal was therefore to try a different ray casting approach using GPGPU computing and OpenCL.

Ray casting is a wide spread technique for creating a two-dimensional image of a three-dimensional scene by casting rays from a camera position through the pixels of an image plane in space on which the final image should be projected. Ray casting has a different run time complexity than traditional rasterization, from which especially large scenes benefit. Ray casting also allows to easily generate images of implicit geometries like CSG models, where the scene is described by boolean combination of volumes. Counters are used in this case to count volume entries and exits until the surface hit is found. To accelerate ray casting several data structures are used to organize the scene more efficiently. One of them are regular grids, where the scene is subdivided into equally sized cubes. Grids are advantageous over other wide-spread data structures like kd trees for being easy and fast in construction, better facilitating dynamic scenes. During ray casting, grids are traversed by individual rays using a 3D variant of the DDA algorithm. Ray packets are guided through the grid slice by slice.

OpenCL is an open and free standard for parallel and general purpose programming targeting heterogeneous platforms. It is most often used to write programs, called kernels, for GPUs. To use OpenCL in an application, an SDK is required to provided the necessary header files and libraries. A typical OpenCL application starts by selecting an available platform and device as well as creating a context and a command queue. Kernels are written in OpenCL C and compiled at run time for the chosen device. Buffers may be created to pass larger blocks of data between the host application and a kernel. Kernels are executed in an n-dimensional range which determines the number of work items which should be executed. The memory model of OpenCL closely resembles modern GPUs by distinguishing between global, local, private (registers) and constant memory.

The existing prototype at the time the internship started is a C++ application built using Visual Studio and Intel's C++ compiler. It heavily uses AVX intrinsics to process ray packets as fast as possible in a SIMD fashion, thus limiting the application to newer processor types. The acceleration structure used is a regular grid. By classifying the grid cells every time a subtraction volume is added, the relevant cells for ray casting can be detected leading to a further reduction of intersection tests. By using subtraction volumes to express complex geometries, the ray casting algorithm has to use counters for volume entries and exists in order to find the implicit surface.

During the internship, several OpenCL ray casters have been developed together with a small OpenCL driver running these casters. Mainly single ray variants where focuses, as they involve no synchronization between individual rays. Several advanced features were added such as double precision support, source file embedding and out of core ray casting. The built infrastructure was finally ported to a new prototype which will be used for public demonstration and provides the base for a final product.

The benchmark results show that the OpenCL implementation can definitely compete with the existing CPU variant with a speedup of two to four on different scenes using single precision. The visual quality of the output is can be considered equal with the CPU double precision implementation. During development, several tools have been tried and evaluated. However, a few problems still remain which could further improve the ray casting performance on GPUs.

In conclusion it can be said that the OpenCL port of the ray caster was successful. Ray casting offers a lot of parallelism due to its parallel nature which can be advantageously used in GPGPU computing. However, we also saw that writing programs for the GPU as not as easy as developing an algorithm on the CPU. Commonly used and well-established concepts like dynamic allocation and communication methods between threads are suddenly unavailable when programming using OpenCL. Different aspects like memory access patterns, branch divergence between threads and register footprint come into play which are usually neglected in traditional software development.

Although the general purpose graphics processing unit becomes more and more capable of executing non-graphical tasks and running algorithms very dissimilar to the ones the a GPU was initially designed for, GPUs are still not general purpose processors. A graphic card remains a highly specialized instruments with its main goal of accelerating computer graphics. Therefore, only a small subset of problems actually benefits from being run on a GPU. Ray casting is fortunately one of them. Although we have also seen, that less independently parallel approaches, like the packet ray caster, which requires a high amount of synchronization, lead to a drastic loss of throughput.

Finally, also the developer tools available for OpenCL are far from being as satisfying as their CPU counterparts. During the internship, Intel's VTune Amplifier has been used for optimizing CPU code and was amazingly helpful. Also debugging C++ code, even if run in multiple threads, is well supported by today's debuggers (like the one integrated in Visual Studio). GPGPU computing is still a young discipline, also for vendors. The provided tools seem to be immature and unstable in several cases (e.g. Intel's OpenCL debugger or AMD's CodeXL). Some tools are also only available on a certain kind of hardware (e.g. NVIDIA's Visual Profiler or AMD's CodeXL). Although consequent improvements and advances are being made, there is still a lot of work to do to make GPU development (especially with OpenCL) as convenient and easy as traditional CPU orientated software development.

Concerning the future of Enlight, the initial planning schedules the project's end to December 2013. During this time, a lot of innovative work has been made which has been submitted to various conferences. Especially the idea of classification was new and proved to largely accelerate the ray casting of scenes described by subtractive volumes. RISC is currently discussing the integration of the ray casting technique developed during Enlight into other customer products, thus bringing the gained knowledge to its application. Furthermore, a follow-up project is in consideration, which may evaluate ray casting using a different but promising new piece of hardware on the high performance market, Intel's many integrated core (MIC) co processor cards. By being designed for massively parallel acceleration using existing software technologies and languages, Intel's MIC architecture does not suffer from the design requirements of a GPU, as it consists of a huge array of conventional Intel x86 cores. Who wants to be annoyed by GPGPU computing then?
