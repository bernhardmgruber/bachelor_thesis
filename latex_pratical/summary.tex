\section{Summary and conclusion}
\label{sec:summary}

The RISC Software GmbH is a limited liability company and part of the software park Hagenberg. It focuses on the practical application of research done in the corresponding RISC institute. In mid 2011, the project Enlight was started with financial help by the governmental Regio 13 program. The goal of Enlight is to create a ray casting solution for interactive visualization of complex geometries consisting. Subtractive manufacturing is the main inspiration for the project. At the start of the internship in April 2013, most functionality was already implemented. The goal was therefore to try a different ray casting approach using GPGPU computing and OpenCL.

Ray casting is a wide spread technique for creating a two-dimensional image of a three-dimensional scene by casting rays from a camera position through the pixels of an image plane in space on which the final image should be projected. Ray casting has a different run time complexity than traditional rasterization, from which especially large scenes benefit. Ray casting also allows to easily generate images of implicit geometries like CSG models, where the scene is described by boolean combination of volumes. Counters are used in this case to count volume entries and exits until the surface hit is found. To accelerate ray casting several data structures are used to organize the scene more efficiently. One of them are regular grids, where the scene is subdivided into equally sized cubes. Grids are advantageous over other wide-spread data structures like kd trees for being easy and fast in construction, better facilitating dynamic scenes. During ray casting, grids are traversed by individual rays using a 3D variant of the DDA algorithm. Ray packets are guided through the grid slice by slice.

OpenCL is an open and free standard for parallel and general purpose programming targeting heterogeneous platforms. It is most often used to write programs, called kernels, for GPUs. To use OpenCL in an application, an SDK is required to provided the necessary header files and libraries. A typical OpenCL application starts by selecting an available platform and device as well as creating a context and a command queue. Kernels are written in OpenCL C and compiled at run time for the chosen device. Buffers may be created to pass larger blocks of data between the host application and a kernel. Kernels are executed in an n-dimensional range which determines the number of work items which should be executed. The memory model of OpenCL closely resembles modern GPUs by distinguishing between global, local, private (registers) and constant memory.

The existing prototype at the time the internship started is a C++ application built using Visual Studio and Intel's C++ compiler. It heavily uses AVX intrinsics to process ray packets as fast as possible in a SIMD fashion, thus limiting the application to newer processor types. The acceleration structure used is a regular grid. By classifying the grid cells every time a subtraction volume is added, the relevant cells for ray casting can be detected leading to a further reduction of intersection tests. By using subtraction volumes to express complex geometries, the ray casting algorithm has to use counters for volume entries and exists in order to find the implicit surface.

During the internship, several OpenCL ray casters have been developed together with a small OpenCL driver running these casters. Mainly single ray variants where focuses, as they involve no synchronization between individual rays. Several advanced features were added such as double precision support, source file embedding and out of core ray casting. The built infrastructure was finally ported to a new prototype which will be used for public demonstration and provides the base for a final product.

The benchmark results show that the OpenCL implementation can definitely compete with the existing CPU variant with a speedup of two to four on different scenes using single precision. The visual quality of the output is can be considered equal with the CPU double precision implementation. During development, several tools have been tried and evaluated. However, a few problems still remain which could further improve the ray casting performance on GPUs.

In conclusion 

GPUs are not general purpose but highly specialized instruments
available tools for NVIDIA are unsatisfying

