\section*{Abstract}

This thesis provides a detailed coverage of the authors internship at RISC Software GmbH. After a short introduction to the company and the Enlight project, the goals of the internship are discussed, which address various aspects of an OpenCL ray caster implementation created during the three months working period.

In succession to the introduction, fundamental topics required for the thesis are covered. These include the principle of ray casting, regular grids as acceleration structure, ray casting implicitly described geometry using boolean subtraction and OpenCL as technology for GPU acceleration.

Before the thesis deepens into the actual implementation, the existing prototype at the start of the internship is analyzed in detail. This includes advanced algorithms to optimize the used ray casting approach.

The primary focus then lies on several OpenCL programs with the goal of reproducing the visual output of the existing CPU implementation by using a GPU with OpenCL. Advantages and difficulties of developing with OpenCL are encountered during the explanations of the implementations.

The final benchmarks of the working OpenCL ray casters are than rounded off by experiences made with various development tools around OpenCL and a discussion of the still remaining problems.