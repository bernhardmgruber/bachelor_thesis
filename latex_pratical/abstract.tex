\section*{Kurzfassung}

Diese Arbeit stellt eine detaillierte Dokumentation über das Berufspraktikum des Autors bei der RISC Software GmbH dar. Nach einer kurzen Einführung in das Unternehmen sowie das Projekt Enlight werden die Ziele des Praktikums erläutert, welche verschiedene Aspekte der Implementierung eines Raycasters mit OpenCL betreffen, der in den drei Praktikumsmonaten erstellt wurde.

Im Anschluss an die Einführung werden grundlegende Themen sowie Basiswissen behandelt, das in späteren Kapiteln benötigt wird. Dazu gehört ein Grundverständnis über Raycasting, reguläre Gitter als Beschleunigungsstruktur, Raycasting implizit beschriebener Geometrie über boolsche Subtraktion und OpenCL als Technologie für GPU Beschleunigung.

Bevor sich die Arbeit in die eigentliche Implementierung vertieft, wird der zum Praktikumsbeginn bestehende Prototyp detailliert analysiert. Dabei wird vor allem auf fortgeschrittene Algorithmen zur Optimierung des eingesetzten Raycasting Verfahrens eingegangen.

Anschließend fokussiert sich die Arbeit auf verschiedenen OpenCL Programme, deren Ziel es ist GPU beschleunigt Bilder in ähnlicher Qualität wie die existierende Implementierung zu erzeugen. Auf Vorteile und Schwierigkeiten während der Implementierung der OpenCL Programme wird ebenfalls eingegangen.

Die abschließenden Laufzeitvergleiche der funktionierenden OpenCL Raycaster mit der existierenden CPU Implementierung werden noch durch Erfahrungen mit Entwicklungswerkzeugen rund um OpenCL und einer kurzen Diskussion über noch offene Probleme abgerundet.

\pagebreak

\section*{Abstract}

This thesis provides a detailed coverage of the authors internship at RISC Software GmbH. After a short introduction to the company and the Enlight project, the goals of the internship are discussed, which address various aspects of an OpenCL ray caster implementation created during the three months working period.

In succession to the introduction, fundamental topics required in further chapters of the thesis are covered. These include the principle of ray casting, regular grids as acceleration structure, ray casting implicitly described geometry using boolean subtraction and OpenCL as technology for GPU acceleration.

Before the thesis deepens into the actual implementation, the existing prototype at the start of the internship is analyzed in detail. This includes advanced algorithms to optimize the used ray casting approach.

The primary focus then lies on several OpenCL programs with the goal of reproducing the visual output of the existing CPU implementation using GPU acceleration. Advantages and difficulties of developing with OpenCL are encountered during the explanations of the implementations.

The final benchmarks of the working OpenCL ray casters are than rounded off by experiences made with various development tools around OpenCL and a discussion of the still remaining problems.
